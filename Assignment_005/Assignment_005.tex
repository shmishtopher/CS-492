%
% @author   Shmish  "shmish90@gmail.com"
% @legal    MIT     "(c) Christopher Schmitt"
%


\documentclass{article}


%
% Document Imports
%

\usepackage{fancyhdr}
\usepackage{extramarks}
\usepackage{amsmath}
\usepackage{amssymb}
\usepackage{amsthm}
\usepackage{amsfonts}
\usepackage{color}
\usepackage{tikz}



%
% Document Configuration
%

\newcommand{\hwAuthor}{Christopher K. Schmitt}
\newcommand{\hwSubject}{CS 492}
\newcommand{\hwSection}{Section 70}
\newcommand{\hwSemester}{Spring 2020}
\newcommand{\hwAssignment}{Assignment 5}


%
% Document Environments
%

\setlength{\headheight}{65pt}
\pagestyle{fancy}
\lhead{\hwAuthor}
\rhead{
  \hwSubject \\
  \hwSection \\
  \hwSemester \\
  \hwAssignment
}

\newenvironment{problem}[1]{
  \nobreak\section*{Problem #1}
}{}


%
% Document Start
%

\begin{document}
  \begin{problem}{1}
    In theory, you would need at least $12$ characters to secure the
    key.  We can find this with the information entropy formula:
    \begin{center}
      \begin{displaymath}
        H = \log_2 N^L
      \end{displaymath}
    \end{center}
    Where $H$ is entropy, $N$ is the number of bits provided per
    symbol, and $L$ is sequence length.  Solving for $L$:
    \begin{center}
      \begin{displaymath}
        96 = \log_2 256^L \implies L = 12 
      \end{displaymath}
    \end{center}
    This is only a theoretic solution, in practice people do not
    achieve required entropy because they do not choose characters
    randomly.  Specific letters and patterns are picked more often
    than others, a secure password requires truly random choices.
    Additionally, if using an an encoding scheme like ASCII, some
    symbols are reserved as control characters and will never show
    up, further reducing entropy.
  \end{problem}

  \begin{problem}{2}
    \begin{itemize}
      \item Windows Hello + Password, Something you are and something you know
      \item Online banking text + password, something you have (your phone) and something you know
      \item Mobile banking fingerprint + password, something you are and something you know
      \item Token generator + fingerprint reader on a laptop, something you have and something you are
      \item Apple Face ID + pin, something you are and something you know
    \end{itemize}
  \end{problem}

  \begin{problem}{3}
    DMZs enable an organization to create a ``perimeter" around their
    LAN.  All incoming and outgoing network traffic passes though
    this layer, while intra-network communication is unhindered.
    This enables the network administrators to provide security while
    keeping internal communications speedy.  It also enables security
    systems to focus their recourses on just the traffic passing
    through this layer.
  \end{problem}

  \begin{problem}{4}
    Pros of Signature Detection:
    \begin{itemize}
      \item Signature based detection systems can be expanded easily by extending the signature database
      \item Signature based systems are very good at stopping known threats
      \item Has a lower false positive rate that anomaly detection
    \end{itemize}
    \noindent
    Pros of Anomaly Detection
    \begin{itemize}
      \item Unlike signature detection, anomaly can prevent unknown attacks
      \item Can be tuned to new applications easily
    \end{itemize}
    A3 and A0 are outside of the threshold, this is anomalous behavior.
  \end{problem}
\end{document}