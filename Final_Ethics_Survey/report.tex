%
% @author   Shmish  "shmish90@gmail.com"
% @legal    MIT     "(c) Christopher Schmitt"
%


\documentclass{article}


%
% Document Imports
%

\usepackage{fancyhdr}
\usepackage{extramarks}
\usepackage{amsmath}
\usepackage{amssymb}
\usepackage{amsthm}
\usepackage{amsfonts}
\usepackage{color}
\usepackage{tikz}



%
% Document Configuration
%

\newcommand{\hwAuthor}{Christopher K. Schmitt}
\newcommand{\hwSubject}{CS 492}
\newcommand{\hwSection}{Section 70}
\newcommand{\hwSemester}{Spring 2020}
\newcommand{\hwAssignment}{Ethics Survey}

\setlength{\headheight}{65pt}
\pagestyle{fancy}
\lhead{\hwAuthor}
\rhead{
  \hwSubject \\
  \hwSection \\
  \hwSemester \\
  \hwAssignment
}


%
% Document Environments
%

\newenvironment{problem}[1]{
  \nobreak\section*{#1}
}{}


%
% Document Start
%

\begin{document}
  \begin{center}
    \large\textbf{The Ethics of End-To-End Cryptosystems}
  \end{center}
  
  Thanks to modern mathematics, digital cryptosystems can be made functionally
  unbreakable.  This revolutionary idea of an unbreakable lock has enabled the
  development of many wonders of the modern world.  Everything from banking to
  blogging to browsing is made possible by these provably invulnerable digital
  locks.  The whole of the internet itself depends on the secure communication
  which these locks provide.  The unbreakable nature of these locks, however,
  presents several ethical and legal predicaments which require both a 
  technical and legal understanding of the problem.

  It's easy to point to situation where unbreakable cryptography is harmful.
  Take for example, the tragedy that took place on December 2, 2015.  A
  domestic terrorist ended the lives of fourteen innocent bystanders, with an
  additional twenty-two injuries.  As part of the investigation, the FBI
  demanded, with a legal warrant, that the phones manufacturer aid in 
  decrypting the phones contents.  The manufacturer refused, citing user
  privacy concerns \cite{Feinstein}.  Society in general agrees that there are
  certain situations where law enforcement would ideally be able to access
  encrypted content.  The ACM code of ethics includes a provision for this with
  its "for the public good" clause \cite{ACM}.  It is impossible, however, to
  to develop a cryptosystem which only the just can use and the partial cannot.
  Therefore, it is necessary to balance the duty to behave responsibly with
  private data and the duty to act ethically with respect to the law.

  Lavabit provides an interesting case study.  Lavabit was a webmail service
  launched in 2014 which advertised itself as a privacy focused alternative to
  Google's popular GMail service.  The Lavabit platform had complied with legal
  warrants in the past involving a user suspected of distributing child
  pornography \cite{LavabitWarrent}.  However, in July of 2013 the FBI ordered
  Lavabit to turn over their SSL keys, which would enable them to monitor all
  of Lavabit's userbase without any sort of notification.  In response, Lavabit
  shut down its operations without turning over the data.  Lavabit founder, 
  Ladar Levison later revealed that he was under gag order not to reveal to the
  public that the FBI had requested the keys \cite{Mullin}.  The government
  argued that The Stored Communication Act allowed the FBI to compel any third
  party to turn over any stored electronic communications \cite{Lavabit}. 
  Levison disagreed, claiming that turning over the SSL keys exposed every one
  of his users' data and violating their fourth amendment protections against
  unreasonable searches and seizures \cite{Lavabit}.  The ACM code of computer
  ethics has several provisions that would address a situation similar in
  nature to Lavabit's.  Specifically, sections 1.2, 1.3, and 1.6 deal with
  privacy and honesty \cite{ACM}.  Section 1.2, ``Avoid harm", explicitly
  mentions: ``Examples of harm include ... unjustified destruction or disclosure
  of information" \cite{ACM}.  In this case, Levison acted to protect his users
  from an obviously unjustified disclosure.  Section 1.3, `Be honest and
  trustworthy" reads: ``A computing professional should be transparent and 
  provide full disclosure of all pertinent system capabilities, limitations, 
  and potential problems to the appropriate parties" \cite{ACM}.  Because
  Levison was under gag order, he was unable to legally disclose that the
  security which he had advertised to his users had been compromised.  In this
  case, shutting down the entire service was the \emph{only} ethical option
  remaining.  Section 1.6, ``Respect Privacy'' is the most pertinent here.  It
  puts the responsibly of maintaining privacy on the computing professional.
  If the onus to defend users' privacy was on Levison, then taking action to
  protect that privacy was ethically necessary.

  In contrast with the Lavabit case, \emph{CARPENTER v. UNITED STATES} offers
  legal arguments which protect a user's data from unreasonable searches as per
  the 4'th amendment, even in cases involving third parties \cite{Carpenter}.
  The facts of the case revolved around the FBI's use of cell tower's
  timestamped logs to confirm that Carpenter, the suspect, was at the scene of
  the crime as it was happening.  Carpenter was convicted on all but one counts
  and sentenced to one-hundred years in prison \cite{Carpenter}.  The sixth
  circuit appeals court held that Carpenter lacked "A reasonable expectation of
  privacy" as cell phone users voluntarily relinquish this data to cell
  carriers and are therefore not entitled to fourth amendment protections.
  Chief Justice Roberts, in writing his opinion, states: ``we determined that 
  the Government - absent a warrant - could not capitalize on such new 
  sense-enhancing technology to explore what was happening within the home''
  \cite{Carpenter}, in reference to \emph{Kyllo v. United States}, where police
  used sophisticated infrared imaging techniques to effectively search a home
  without first obtaining a legal warrant to do so.  Roberts also states: ``
  ... police officers must generally obtain a warrant before searching the 
  contents of a phone'' \cite{Carpenter}.  Roberts held that: ``A person does 
  not surrender all Fourth Amendment protection by venturing into the public 
  sphere. To the contrary, ``what [one] seeks to preserve as private, even in 
  an area accessible to the public, may be constitutionally protected.'' Katz, 
  389 U. S., at 351–352'' \cite{Carpenter}.  Roberts recognized the threat to
  privacy that such technology posed, stating: ``... cell phone tracking is 
  remarkably easy, cheap, and efficient compared to traditional investigative 
  tools. With just the click of a button, the Government can access each 
  carrier's deep repository of historical location information at practically 
  no expense'' \cite{Carpenter}, and: ``In fact, historical cell-site records 
  present even greater privacy concerns than GPS monitoring ...'' 
  \cite{Carpenter}.  This supreme court decision, held 5-4 in favor of
  Carpenter, established that third party doctrine cannot simply be applied
  mechanically.  Context matters and the users privacy is of upmost importance
  when designing a service that utilizes user data of any kind.  This applies
  to data that is not strictly own by the user.  In this case, cell records
  where not supplied by the user, but created by the carrier.

  Despite the ruling of the courts in Carpenter v. United States, service
  providers big and small can still be (and have been) approached by law
  enforcement and been compelled to surrender documents to investigators under
  gag order and without a legal warrant.  In face of this, the onus is on the
  developers to take appropriate action to defend user privacy.  In
  applications where the service provider does not need to view and access
  communications (like messaging services), techniques like end to end
  encryption can be used.  In end to end encryption schemes, the service 
  provider does not hold any keys, they simply act as a proxy between the
  senders and receivers of messages.  Because the service provider does not
  hold any of the keys required to decrypt any message, trying to compel a
  service provider to aid in any surveillance actions is pointless.  No
  government agency would be able to use a warrant, legal or otherwise, to spy
  on people in an illegal manner.  The government can still compel an
  individual to hand over their private keys, but this is entirely legal.  This
  is the closest to an ideal situation that can realistically be achieved.  Law
  enforcement would still be able to access necessary data \emph{sometimes}, 
  but trying to perform any mass surveillance over the service would become
  impractical.

  \begin{center}
    \usetikzlibrary{arrows}
    \tikzstyle{int}=[draw, fill=blue!20, minimum size=2em]
    \tikzstyle{init} = [pin edge={to-,thin,black}]

    \begin{tikzpicture}[node distance=3.5cm,auto,>=latex']
      \node [int] (Service) {Service};
      \node [int, left of=Service] (Alice) {Alice};
      \node [int, right of=Service] (Bob) {Bob};

      \path[<-] (Bob) edge node {$\{M\}_B$} (Service);
      \path[<-] (Alice) edge node {$[B_{public}]_B$} (Service);
      \path[->] (Alice) edge node[below] {$\{M\}_B$} (Service);
    \end{tikzpicture}
  \end{center}

  The above figure demonstrates states how an (unsecure) service might be
  structured to prevent the service provider from being able to know the
  contents of a message.  If Alice wished to send a message to bob, she first
  retrieves his public key from the service.  She then encrypts the message
  using bobs public key.  The service then simply forwards the message onto
  Bob, who can decrypt the message using his private key.  Because everyone
  maintains their own keys, it is impossible for the service provider to read
  any of the communications between Alice and Bob.  Note that it is not
  impossible for their communications to be read, This model is vulnerable to
  a man-in-the-middle attack.  But is is impossible to ask the service provider
  who only maintains a list of public keys to read any message sent to Bob.
  A model like this allows the service provider to protect users' privacy while
  maintaining a certain legal standard.
  
  \bibliographystyle{unsrt}
  \bibliography{refrences}
\end{document}