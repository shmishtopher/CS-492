%
% @author   Shmish  "shmish90@gmail.com"
% @legal    MIT     "(c) Christopher Schmitt"
%


\documentclass{article}


%
% Document Imports
%

\usepackage{fancyhdr}
\usepackage{extramarks}
\usepackage{amsmath}
\usepackage{amssymb}
\usepackage{amsthm}
\usepackage{amsfonts}
\usepackage{color}
\usepackage{tikz}



%
% Document Configuration
%

\newcommand{\hwAuthor}{Christopher K. Schmitt}
\newcommand{\hwSubject}{CS 492}
\newcommand{\hwSection}{Section 70}
\newcommand{\hwSemester}{Spring 2020}
\newcommand{\hwAssignment}{Assignment 4}


%
% Document Environments
%

\setlength{\headheight}{65pt}
\pagestyle{fancy}
\lhead{\hwAuthor}
\rhead{
  \hwSubject \\
  \hwSection \\
  \hwSemester \\
  \hwAssignment
}

\newenvironment{problem}[1]{
  \nobreak\section*{Problem #1}
}{}


%
% Document Start
%

\begin{document}
  \begin{problem}{1}
    Design a secure mutual authentication protocol based on a shared 
    symmetric key.  We also want to establish a session key, and we 
    want perfect forward secrecy.  Solve for a protocol that can 
    establish this in 2 to 3 messages

    \begin{center}
      \begin{tikzpicture}
        \node (Alice) at (0, 1) {Alice, K};
        \node (Bob) at (7, 1) {Bob, K};

        \draw [->] (1, 2) -- (6, 2) node[midway, above] {``I'm Alice'', $R_{A}$};
        \draw [<-] (1, 1) -- (6, 1) node[midway, above] {$E(\text{``I'm Bob''}, R_{A}, R_{B}, K)$};
        \draw [->] (1, 0) -- (6, 0) node[midway, above] {$E(\text{``I'm Alice''}, R_{A}, R_{B}, SK, K)$};
      \end{tikzpicture}
    \end{center}
  \end{problem}

  \begin{problem}{2}
    Draw the sequence of an attack Trudy can use to convince Bob that 
    she is Alice.

    \begin{center}
      \begin{tikzpicture}
        \node (Trudy) at (-1, 3) {Trudy};
        \node (Bob) at (6, 3) {Bob};

        \draw [->] (0, 5) -- (5, 5) node[midway, above] {``I'm Alice'', $R + 1$};
        \draw [<-] (0, 4) -- (5, 4) node[midway, above] {$E(R + 1, K_{AB})$};
        \draw [->] (0, 3) -- (5, 3) node[midway, above] {``I'm Alice'', $R$};
        \draw [<-] (0, 2) -- (5, 2) node[midway, above] {$E(R, K_{AB})$};
        \draw [->] (0, 1) -- (5, 1) node[midway, above] {$E(R + 1, K_{AB})$};
      \end{tikzpicture}
    \end{center}
  \end{problem}

  \begin{problem}{3}
    Does Alice authenticate Bob?  Justify your answer.
    \bigbreak
    Alice is able to authenticate Bob because in order to compute K,
    we must have S.  S is encrypted by Alice using Bob's public key
    so only Bob is able to decrypt S and compute K.
    \bigbreak\noindent
    Does Bob authenticate Alice?  Justify your answer.
    \bigbreak
    No, Bob is unable to authenticate Alice.  There is no operation
    that Alice is required to perform that only she can compute.
    Trudy has all the information she needs to imitate Alice since
    anyone can encrypt S for Bob.  If CLNT is stored securely
    however, Trudy shouldn't be able to hijack anything specific to
    Alice maintained by Bob.
  \end{problem}

  \begin{problem}{4}
    Suppose that nonces RA and RB are removed from the protocol and 
    K=h(S).  What effect if any does that have on the security of the
    protocol?
    \bigbreak
    Removing the nonces in this protocol leaves the communication
    vulnerable to replay attacks.  Without the nonces, the first
    three messages are identical between instances.
    \bigbreak\noindent
    Suppose we change message 4 to HMAC(msgs, SRVR, K).  What effect, 
    if any, does this have on the security of the authentication 
    protocol?
    \bigbreak
    Replacing the hash with the HMAC would let Alice know if some third
    party has tampered with the messages, since computing the HMAC
    requires knowing the key.
    \bigbreak\noindent
    Suppose that we change message three to $\{S\}_{Bob}$, h(msgs, CLNT, 
    K)  What effect, if any, does this have on the security of the 
    authentication protocol?
    \bigbreak
    This should have no effect on the security of the protocol.  A
    hash is a one way function, so no information can be extracted
    from it.  Bob can still verify it by computing the hash himself
    and comparing it.
  \end{problem}

  \begin{problem}{5}
    Why can Alice not remain anonymous when requesting a TGT from the
    KDC?
    \bigbreak
    Alice must be authenticated, and her identity is contained in the
    TGT, so she can't remain anonymous.
    \bigbreak\noindent
    Why can Alice remain anonymous in the sense of not needing to use 
    her private key when requesting a ticket to Bob (what does she 
    use instead and why is this sufficient)?
    \bigbreak
    Alice has her TGT, which contains her identity behind a key known
    only to the KCD.
    \bigbreak\noindent
    Why can Alice remain anonymous (not needing her private key) when 
    she sends the ``ticket to Bob'' to Bob?
    \bigbreak
    The ticket to bob is issued by the KCD, which used Alice's TGT to
    authenticate her.  If Bob trusts the KCD, he can trust the ticket
    to bob without needing to authenticate the person on the other
    end because the KCD has already done that.

  \end{problem}
\end{document}